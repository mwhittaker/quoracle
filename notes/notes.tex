\documentclass[12pt]{article}
\usepackage[margin=1in]{geometry}
\usepackage{pervasives}

\begin{document}
\begin{center}
  {\Large Quorum Systems}

  \today{}
\end{center}

{\section{Quorum Systems}
Given a set $X = \set{x_1, \ldots, x_n}$, a \defword{quorum system} over $X$ is
a set $Q = \set{q_1, \ldots, q_m}$ of subsets of $X$, called \defword{quorums},
such that every pair of quorums intersect. That is, for every $q_1, q_2 \in
Q$, $q_1 \cap q_2 \neq \emptyset$.
%
A quorum system $Q$ is a \defword{coterie} if there does not exist quorums
$q_1, q_2 \in Q$ such that $q_1 \subset q_2$. In other words, a coterie is a
quorum system that does not contain some quorum $q_1$ that is a strict subset
of some other quorum $q_2$.
%
Let $P, Q$ be two quorum systems over the same set $X$. $P$ \defword{dominates}
$Q$, denoted $P > Q$, if $P \neq Q$ and for every $q \in Q$, there exists some
$p \in P$ such that $p \subseteq q$. A quorum system $Q$ is \defword{dominated}
if there exists some quorum system $P$ that dominates it.

We can associate every quorum system $Q$ with a monotone boolean function
$f_Q$. For example, the majority quorum system $Q = \set{\set{a, b}, \set{b,
c}, \set{a, c}}$ corresponds to the function $f_Q = ab + bc + ac$. The prime
implicants of the boolean function correspond to the minimal sets of the quorum
system. We say $f \leq g$ if for every $\vec{x}$, $f(\vec{x}) \implies
g(\vec{x})$. In other words, $f \leq g$ if $\setst{\vec{x}}{g(\vec{x})}
\subseteq \setst{\vec{x}}{f(\vec{x})}$. Consider two quorum systems $Q_1$ and
$Q_2$, $Q_1$ dominates $Q_2$ if and only if $f_{Q_2} < f_{Q_1}$.

Let the dual $\dual{f}$ of $f$ be $\dual{f}(x) = \bar{f}(\bar{x})$. $\dual{f}$
is the function we get if we swap and with or. Note that $\dual{f}$ corresponds
to the sets that intersect every set in $f$. We say a monotone function $f$
is \defword{dual-minor} if $f \leq \dual{f}$, \defword{dual-major} if $f \geq
\dual{f}$, and \defword{self-dual} if $f = \dual{f}$. A function $f$
corresponds to a quorum system if and only if $f$ is dual-minor.  A quorum
system $Q$ is non-dominated if and only if $f_Q$ is self-dual. Every
non-dominated coterie can be represented as a composition of the simple
majority function (with duplicates). Every coterie can be represented by a
composition of and, or, and majority. \cite{ibaraki1993theory} also talks about
when a function can be decomposed without duplicates as well as other things
involving decomposition. \cite{neilsen1991general} talks about composition as
well, and even includes examples for things like grids and trees.

Let $\sigma: Q \to [0, 1]$ be a discrete probability distribution over the
quorums of $Q$ (i.e., $\sum_{q \in Q} \sigma(q) = 1$). We call $\sigma$ a
\defword{strategy}. Intuitively, $\sigma$ is a strategy to pick quorums at
random. We have the following definitions.
\begin{align*}
  l_\sigma(x) &\defeq \sum_{\setst{q \in Q}{x \in q}} \sigma(q) \\
  L_\sigma(Q) &\defeq \max_{x \in X} l_\sigma(x) \\
  L(Q)        &\defeq \min_\sigma L_\sigma(Q)
\end{align*}
$l_\sigma(x)$ is the load on $x$ given some strategy $\sigma$. $L_\sigma(Q)$ is
the load on most loaded element $x$. $L(Q)$ is the \defword{load} of the best
possible strategy. Intuitively, the lower the load of a quorum system, the
higher the throughput it can support.

\begin{example}
  Consider the majority quorum system $Q = \set{\set{a, b}, \set{a, c}, \set{b,
  c}}$ on elements $X = \set{a, b, c}$. Let $\sigma(\set{a, b}) =
  \sigma(\set{a, c}) = 0.5$ and $\sigma(\set{b, c}) = 0$.
  \begin{align*}
    l_\sigma(a) &= \sigma(\set{a, b}) + \sigma(\set{a, c}) = 0.5 + 0.5 = 1 \\
    l_\sigma(b) &= \sigma(\set{a, b}) + \sigma(\set{b, c}) = 0.5 + 0 = 0.5 \\
    l_\sigma(c) &= \sigma(\set{a, c}) + \sigma(\set{b, c}) = 0.5 + 0 = 0.5 \\
    L_\sigma(Q) &= \max(1, 0.5, 0.5) = 1
  \end{align*}
  The load of $Q$ with respect to $\sigma$ is 1, but the load $L(Q)$ is not 1
  because $\sigma$ is not an optimal strategy. If we instead choose
  $\sigma_{\text{opt}}(-) = \frac{1}{3}$, the load is reduced to $\frac{2}{3}$
  which is optimal.
\end{example}

\cite{naor1998load} shows how to use a linear program to compute load.
\cite{naor1998load} also proves that if a quorum system dominates another, it
has lower or equal load. This shows that there always is a non-dominated
coterie that has the lowest possible load. If we're trying to optimize for
load, this tells us that we limit ourselves to non-dominated coteries.

The \defword{resilience} or \defword{fault tolerance} of a quorum system $Q$ is
the largest number $f$ such that for every subset $F \subseteq X$ with $|F| =
f$, there still exists some quorum $q \in Q$ such that $q \cap F = \emptyset$.
Intuitively, a quorum system has fault tolerance $f$ if we can fail an
arbitrary set of $f$ elements and still have some quorum left.
}
{\section{Read-Write Quorum Systems}
Given a set $X = \set{x_1, \ldots, x_n}$, a \defword{read-write quorum system}
over $X$ is a pair of sets $Q = (R, W)$ of subsets of $X$ such that every $r
\in R$ intersects every $w \in W$. The elements $r \in R$ are called
\defword{read quorums}, and the elements $w \in W$ are called \defword{write
quorums}. Read-write quorum systems are also called bicoteries.
%
$(R, W)$ is a coterie if both $R$ and $W$ are minimal. $(R_1, W_1)$ dominates
$(R_2, W_2)$ if $R_1$ dominates $R_2$ and $W_1$ dominates $W_2$. Again, we can
view a read-write quorum system $(R, W)$ as a pair of monotone functions $f_R$
and $f_W$. $(f_R, f_W)$ is a read-write quorum sytem if $f_R \leq \dual{f_W}$.
It is a non-dominated coterie if $f_R = \dual{f_W}$. Thus, to generate a
non-dominated bicoterie, let $f_R$ be an arbitrary set of quorums and let $f_W$
be $\dual{f_R}$.

Let $\sigma_R: R \to [0, 1]$ and $\sigma_W: W \to [0, 1]$ be a discrete
probability distribution over the read and write quorums of $Q$. We call
$\sigma$ a \defword{strategy}. Let $0 \leq p_r \leq 1$ be the probability of
performing a read and $p_w = 1 - p_r$ be the probability of performing a write.
We have the following definitions.
\begin{align*}
  l_{\sigma,p_r,p_w}(x)
    &\defeq p_r \cdot \parens*{\sum_{\setst{r \in R}{x \in r}} \sigma_R(r)} +
            p_w \cdot \parens*{\sum_{\setst{w \in W}{x \in w}} \sigma_W(w)} \\
  L_{\sigma,p_r,p_w}(Q) &\defeq \max_{x \in X} l_{\sigma,p_r,p_w}(x) \\
  L_{p_r,p_w}(Q)        &\defeq \min_\sigma L_{\sigma,p_r,p_w}(Q)
\end{align*}
$l_{\sigma,p_r,p_w}(x)$ is the load on $x$ given some strategy $\sigma$ and
some workload $p_r,p_w$. $L_{\sigma,p_r,p_w}(Q)$ is the load on most loaded
element $x$. $L{p_r,p_w}(Q)$ is the \defword{load} of the best possible
strategy.

The \defword{read resilience} or \defword{read fault tolerance} of a quorum
system $Q$ is the largest number $f_r$ such that for every subset $F \subseteq
X$ with $|F| = f_r$, there still exists some read quorum $r \in R$ such that $r
\cap F = \emptyset$.
%
The \defword{write resilience} or \defword{write fault tolerance} of a quorum
system $Q$ is the largest number $f_w$ such that for every subset $F \subseteq
X$ with $|F| = f_w$, there still exists some write quorum $w \in W$ such that $w
\cap F = \emptyset$.
%
The \defword{resilience} or \defword{fault tolerance} of a quorum system is the
minimum of its read resilience and write resilience. Intuitively, a quorum
system has read fault tolerance $f_r$ if we can fail an arbitrary set of $f_r$
elements and still have some read quorum left; a quorum system has write fault
tolerance $f_w$ if we can fail an arbitrary set of $f_w$ elements and still
have some write quorum left; and a quorum system has fault tolerance $f$ if we
can fail an arbitrary set of $f$ elements and still have some read quorum and
some write quorum left.
}
{\input{sections/recursive_quorum_systems.tex}}

\TODO[michael]{Prove that for every read-write quorum system, there exists a coterie with at least as good load.}
\TODO[michael]{Prove that if P dominates Q, then P has equal or lower load.}
\TODO[michael]{The above shows that the optimal load quorum system is a non-dominated coterie. Maybe this is useful? We can generate every NDC?}
\TODO[michael]{Understand how domination relates to subsumption.}
\TODO[michael]{Understand dual-major, dual-minor, and self-dual.}
\TODO[michael]{Extend these notions to read-write quorums.}
\TODO[michael]{Enumerate all read-write quorums on four nodes and see the ones we can and can't subsume.}
\TODO[michael]{Prove that if I have the read quorums R, then the write quorums are bar(R).}


\bibliographystyle{plain}
\bibliography{references}
\end{document}
