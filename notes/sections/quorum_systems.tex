\section{Quorum Systems}
Given a set $X = \set{x_1, \ldots, x_n}$, a \defword{quorum system} over $X$ is
a set $Q = \set{q_1, \ldots, q_m}$ of subsets of $X$, called \defword{quorums},
such that every pair of quorums intersect. That is, for every $q_1, q_2 \in
Q$, $q_1 \cap q_2 \neq \emptyset$.
%
A quorum system $Q$ is a \defword{coterie} if there does not exist quorums
$q_1, q_2 \in Q$ such that $q_1 \subset q_2$. In other words, a coterie is a
quorum system that does not contain some quorum $q_1$ that is a strict subset
of some other quorum $q_2$.
%
Let $P, Q$ be two quorum systems over the same set $X$. $P$ \defword{dominates}
$Q$, denoted $P > Q$, if $P \neq Q$ and for every $q \in Q$, there exists some
$p \in P$ such that $p \subseteq q$. A quorum system $Q$ is \defword{dominated}
if there exists some quorum system $P$ that dominates it.

We can associate every quorum system $Q$ with a monotone boolean function
$f_Q$. For example, the majority quorum system $Q = \set{\set{a, b}, \set{b,
c}, \set{a, c}}$ corresponds to the function $f_Q = ab + bc + ac$. The prime
implicants of the boolean function correspond to the minimal sets of the quorum
system. We say $f \leq g$ if for every $\vec{x}$, $f(\vec{x}) \implies
g(\vec{x})$. In other words, $f \leq g$ if $\setst{\vec{x}}{g(\vec{x})}
\subseteq \setst{\vec{x}}{f(\vec{x})}$. Consider two quorum systems $Q_1$ and
$Q_2$, $Q_1$ dominates $Q_2$ if and only if $f_{Q_2} < f_{Q_1}$.

Let the dual $\dual{f}$ of $f$ be $\dual{f}(x) = \bar{f}(\bar{x})$. $\dual{f}$
is the function we get if we swap and with or. Note that $\dual{f}$ corresponds
to the sets that intersect every set in $f$. We say a monotone function $f$
is \defword{dual-minor} if $f \leq \dual{f}$, \defword{dual-major} if $f \geq
\dual{f}$, and \defword{self-dual} if $f = \dual{f}$. A function $f$
corresponds to a quorum system if and only if $f$ is dual-minor.  A quorum
system $Q$ is non-dominated if and only if $f_Q$ is self-dual. Every
non-dominated coterie can be represented as a composition of the simple
majority function (with duplicates). Every coterie can be represented by a
composition of and, or, and majority. \cite{ibaraki1993theory} also talks about
when a function can be decomposed without duplicates as well as other things
involving decomposition. \cite{neilsen1991general} talks about composition as
well, and even includes examples for things like grids and trees.

Let $\sigma: Q \to [0, 1]$ be a discrete probability distribution over the
quorums of $Q$ (i.e., $\sum_{q \in Q} \sigma(q) = 1$). We call $\sigma$ a
\defword{strategy}. Intuitively, $\sigma$ is a strategy to pick quorums at
random. We have the following definitions.
\begin{align*}
  l_\sigma(x) &\defeq \sum_{\setst{q \in Q}{x \in q}} \sigma(q) \\
  L_\sigma(Q) &\defeq \max_{x \in X} l_\sigma(x) \\
  L(Q)        &\defeq \min_\sigma L_\sigma(Q)
\end{align*}
$l_\sigma(x)$ is the load on $x$ given some strategy $\sigma$. $L_\sigma(Q)$ is
the load on most loaded element $x$. $L(Q)$ is the \defword{load} of the best
possible strategy. Intuitively, the lower the load of a quorum system, the
higher the throughput it can support.

\begin{example}
  Consider the majority quorum system $Q = \set{\set{a, b}, \set{a, c}, \set{b,
  c}}$ on elements $X = \set{a, b, c}$. Let $\sigma(\set{a, b}) =
  \sigma(\set{a, c}) = 0.5$ and $\sigma(\set{b, c}) = 0$.
  \begin{align*}
    l_\sigma(a) &= \sigma(\set{a, b}) + \sigma(\set{a, c}) = 0.5 + 0.5 = 1 \\
    l_\sigma(b) &= \sigma(\set{a, b}) + \sigma(\set{b, c}) = 0.5 + 0 = 0.5 \\
    l_\sigma(c) &= \sigma(\set{a, c}) + \sigma(\set{b, c}) = 0.5 + 0 = 0.5 \\
    L_\sigma(Q) &= \max(1, 0.5, 0.5) = 1
  \end{align*}
  The load of $Q$ with respect to $\sigma$ is 1, but the load $L(Q)$ is not 1
  because $\sigma$ is not an optimal strategy. If we instead choose
  $\sigma_{\text{opt}}(-) = \frac{1}{3}$, the load is reduced to $\frac{2}{3}$
  which is optimal.
\end{example}

\cite{naor1998load} shows how to use a linear program to compute load.
\cite{naor1998load} also proves that if a quorum system dominates another, it
has lower or equal load. This shows that there always is a non-dominated
coterie that has the lowest possible load. If we're trying to optimize for
load, this tells us that we limit ourselves to non-dominated coteries.

The \defword{resilience} or \defword{fault tolerance} of a quorum system $Q$ is
the largest number $f$ such that for every subset $F \subseteq X$ with $|F| =
f$, there still exists some quorum $q \in Q$ such that $q \cap F = \emptyset$.
Intuitively, a quorum system has fault tolerance $f$ if we can fail an
arbitrary set of $f$ elements and still have some quorum left.
